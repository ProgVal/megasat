\documentclass{article}
\usepackage{mathpartir}
\usepackage[utf8]{inputenc}
\usepackage[francais]{babel}
\usepackage[top=30mm, bottom=30mm, left=15mm, right=15mm]{geometry}
\usepackage{amsmath}
\usepackage{color}
\newcommand{\non}[1]{\overline{#1}}
\newcommand{\varv}[1]{x_{#1}}
\newcommand{\varf}[1]{\non{\varv{#1}}}
\newcommand{\cl}[1]{\mathtt{C_{#1}}:~}
\newcommand{\preuve}[1]{\mathtt{\Pi_{#1}}}
\title{$\sqrt{-1}$ love you !}
\author{Marc \textsc{Chevalier}\\ Thomas \textsc{Pellissier Tanon}}
\begin{document}
\maketitle
Preuve de résolution pour la clause :
$$\varf{1} \lor \varv{2}\enspace .$$ 

\begin{mathpar} 
 \inferrule{ 
 \mathbf{\varv{3}} \lor \varv{2} \lor \varf{1}  \and 
 \mathbf{\varf{3}} \lor \varv{2} \lor \varf{1} 
 } { 
 \varf{1} \lor \varv{2} } 
 \end{mathpar}\\ 
 
 
\end{document}
